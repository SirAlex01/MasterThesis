% appendix_llm_filter.tex
\section{LLM filtering prompt and output schema}\label{app:llm-filter}

This appendix reports the full XML prompt used to filter and re-rank candidate cognates, together with the expected JSON output format and a minimal example.
The content below preserves the exact field names as implemented (including ``greek\_correspondance'').

\subsection{Full XML prompt}

\begin{lstlisting}[style=mystyle, language=XML, breaklines=true]
<root>
  <linguistic_principles>
    Linear B is a syllabic script used for writing Mycenaean Greek (1450-1200 BCE).
    When matching Linear B words with their Greek cognates, the following fundamental principles of historical linguistics must be observed:
    <principle_1>
      1. Distributional Similarity of Matching Characters: Corresponding characters tend to appear in similar contexts in both languages. Their phonetic environments should show consistent correspondence patterns.
    </principle_1>
    <principle_2>
      2. Monotonic Character Mapping within Cognates: Matching cognates rarely exhibit character reordering. The alignment between Linear B and ancient Greek must preserve the relative ordering of phonemes, with transformations following systematic phonetic rules.
    </principle_2>
    <principle_3>
      3. Structural Sparsity of Cognate Mapping: As documented in historical linguistics, cognate matches are mostly one-to-one, since both words are derived from the same proto-origin. Avoid unjustified many-to-many correspondences.
    </principle_3>
    <principle_4>
      4. Significant Cognate Overlap Within Related Languages: The derived vocabulary mapping should have sufficient coverage for lost language cognates (Linear B). Prioritize correspondences that reinforce systematic patterns already observed.
    </principle_4>
  </linguistic_principles>

  <task_instructions>
    <task_description>
      Analyze the provided Linear B word and propose the most plausible Ancient Greek cognates,
      rigorously applying the four principles of historical linguistics listed above. 
      Limit your response to a maximum of 3 cognates, selecting only those with the highest linguistic probability.
    </task_description>
    <rule_0>OUTPUT: Maximum 3 cognates separated by commas.</rule_0>
    <rule_1>CRUCIAL: Use ONLY these characters in your output: fhαβγδεζηθικλμνξοπρςστυφχψω. do NOT use ANY OTHER LATIN CHARACTER, ACCENTS, SUBSCRIPT IOTA, SPIRITS.</rule_1>
    <rule_2>CRUCIAL: Consider that our proposed cognates are just proposals of words with similar root in the Homeric scripts. DO NOT RELY ON THEM FOR FINDING THE COGNATES, FIND THEM YOURSELF FROM THE ENTIRE ANCIENT GREEK CORPUS!!!For each proposed correspondence, verify that it is consistent with Principle 1 (Distributional Similarity)</rule_2>
    <rule_3>Every transformation must respect the relative ordering of phonemes, as required by Principle 2 (Monotonic Mapping)</rule_3>
    <rule_4>Prioritize one-to-one correspondences between phonemes, as indicated by Principle 3 (Structural Sparsity)</rule_4>
    <rule_5>Verify that your proposals align with correspondence patterns already documented between Linear B and ancient Greek (Principle 4)</rule_5>
    <rule_6>For syllabograms with 'ha', insert 'h' in the corresponding position in the Ancient Greek sequence</rule_6>
    <rule_7>For syllabograms containing 'digamma' (w), insert 'f' in the corresponding position</rule_7>
    <rule_8>QUALITY CHECK: If you have uncertainties about a cognate, DO NOT include it in the output</rule_8>
  </task_instructions>

  <phonological_mapping>
    <mapping_entry_0>
      <lb_consonant>k</lb_consonant>
      <greek_correspondance_0>κ</greek_correspondance_0>
      <greek_correspondance_1>χ</greek_correspondance_1>
      <greek_correspondance_2>γ</greek_correspondance_2>
      <greek_correspondance_3>ξ</greek_correspondance_3>
      <notes>Linear B does not distinguish between κ, χ, and γ. May become ξ, usually when followed by a syllabogram starting with s. Principle 3: maintain one-to-one correspondences where possible.</notes>
    </mapping_entry_0>
    <mapping_entry_1>
      <lb_consonant>p</lb_consonant>
      <greek_correspondance_0>π</greek_correspondance_0>
      <greek_correspondance_1>φ</greek_correspondance_1>
      <greek_correspondance_2>ψ</greek_correspondance_2>
      <notes>Represents π and φ undifferentiated. May become ψ, usually when followed by a syllabogram starting with s. Principle 1: consider phonetic context to determine the correct correspondence.</notes>
    </mapping_entry_1>
    <mapping_entry_2>
      <lb_consonant>q</lb_consonant>
      <greek_correspondance_0>β</greek_correspondance_0>
      <greek_correspondance_1>π</greek_correspondance_1>
      <greek_correspondance_2>κ</greek_correspondance_2>
      <greek_correspondance_3>γ</greek_correspondance_3>
      <greek_correspondance_4>ψ</greek_correspondance_4>
      <greek_correspondance_5>ξ</greek_correspondance_5>
      <notes>Typically represents labiovelar consonants. May become ξ or ψ, usually when followed by a syllabogram starting with s. Principle 2: ensure transformations follow documented sound change patterns.</notes>
    </mapping_entry_2>
    <mapping_entry_3>
      <lb_consonant>r</lb_consonant>
      <greek_correspondance_0>ρ</greek_correspondance_0>
      <greek_correspondance_1>λ</greek_correspondance_1>
      <notes>Represents both liquids ρ and λ. Apply Principle 4 to determine the most likely correspondence in this specific context.</notes>
    </mapping_entry_3>
    <mapping_entry_4>
      <lb_consonant>s</lb_consonant>
      <greek_correspondance_0>$\sigma$</greek_correspondance_0>
      <greek_correspondance_1>ς</greek_correspondance_1>
      <notes>Apply Principles 1 and 4 to determine whether word-final 'ς' or word-internal 'σ' is appropriate.</notes>
    </mapping_entry_4>
    <mapping_entry_5>
      <lb_consonant>t</lb_consonant>
      <greek_correspondance_0>τ</greek_correspondance_0>
      <greek_correspondance_1>θ</greek_correspondance_1>
    </mapping_entry_5>
    <mapping_entry_6>
      <lb_consonant>d</lb_consonant>
      <greek_correspondance_0>δ</greek_correspondance_0>
    </mapping_entry_6>
    <mapping_entry_7>
      <lb_consonant>m</lb_consonant>
      <greek_correspondance_0>μ</greek_correspondance_0>
    </mapping_entry_7>
    <mapping_entry_8>
      <lb_consonant>n</lb_consonant>
      <greek_correspondance_0>ν</greek_correspondance_0>
    </mapping_entry_8>
    <mapping_entry_9>
      <lb_consonant>z</lb_consonant>
      <greek_correspondance_0>ζ</greek_correspondance_0>
    </mapping_entry_9>
    <mapping_entry_10>
      <lb_consonant>j</lb_consonant>
      <greek_correspondance_0>ι</greek_correspondance_0>
      <notes>Represents yod (semivowel). Principle 1: examine distributional patterns to determine its most likely Greek reflex.</notes>
    </mapping_entry_10>
    <mapping_entry_11>
      <lb_consonant>w</lb_consonant>
      <greek_correspondance_0>f</greek_correspondance_0>
      <notes>Represents digamma (ϝ). Principle 3: mark with 'f' to maintain structural sparsity in the mapping.</notes>
    </mapping_entry_11>
    <mapping_entry_12>
      <lb_consonant>h</lb_consonant>
      <greek_correspondance_0>h</greek_correspondance_0>
    </mapping_entry_12>
    <mapping_entry_13>
      <lb_consonant>y</lb_consonant>
      <greek_correspondance_0>ι</greek_correspondance_0>
      <greek_correspondance_1>ει</greek_correspondance_1>
      <notes>In complex syllabograms. Principle 2: ensure monotonic mapping when resolving to ι or ει.</notes>
    </mapping_entry_13>
    <mapping_entry_14>
      <lb_consonant>ph</lb_consonant>
      <greek_correspondance_0>φ</greek_correspondance_0>
    </mapping_entry_14>
  </phonological_mapping>

  <vowel_transformations>
    <vowel_a>Generally stable as α. Principle 1: examine distributional context for potential contractions.</vowel_a>
    <vowel_e>Typically corresponds to ε, but may represent η in certain positions. Apply Principle 4 based on documented patterns.</vowel_e>
    <vowel_i>Corresponds to ι. Apply Principle 2 to maintain monotonic mapping with any adjacent vowels.</vowel_i>
    <vowel_o>Usually represents ο, but may represent ω in certain positions, especially at the end of the words in verbs. Apply Principle 3 for sparsity in mapping.</vowel_o>
    <vowel_u>Corresponds to υ. Apply Principles 1 and 4 to determine the most likely correspondence.</vowel_u>
  </vowel_transformations>

  <examples>
    <example>
      <input>a-e-ti-to</input>
      <output>αεθιστος, εθιζω</output>
      <principles_applied>Principle 1: t → θ correspondence follows consistent patterns. Principle 2: Preserves monotonic ordering of phonemes. Principle 3: Maintains one-to-one mapping between Linear B and Greek sounds.</principles_applied>
    </example>
    <example>
      <input>a-di-nwa-ta</input>
      <output>αδινfατας</output>
      <principles_applied>Principle 2: Preserves monotonic mapping in nw → νf sequence. Principle 4: Consistent handling of digamma (w → f) reinforces observed patterns.</principles_applied>
    </example>
    <example>
      <input>e-ma-ha</input>
      <output>ερμαhαι, ερμαhας</output>
      <principles_applied>Principle 1: Distributional similarity used to infer ρ. Principle 3: Structural sparsity maintained with one-to-one sound mapping. Principle 4: Consistent h-series representation.</principles_applied>
    </example>
    <example>
      <input>ko-no-so</input>
      <output>κνωσος</output>
      <principles_applied>All four principles applied: Principle 1: k/o sound correspondence. Principle 2: Monotonic ordering preserved. Principle 3: One-to-one mapping. Principle 4: Aligns with known patterns for toponyms.</principles_applied>
    </example>
    <example>
      <input>wo-no-qe-wa</input>
      <output>fονοκεfα</output>
      <principles_applied>Principle 1: w → f and q → κ correspondences follow consistent patterns. Principle 2: Monotonic ordering preserved. Principle 4: Reinforces observed digamma patterns.</principles_applied>
    </example>
    <example>
      <input>wo-ro-ki-jo-ne-jo</input>
      <output>fοργιονειος, οργεωνες</output>
      <principles_applied>Principle 1: Distributional similarity in vowel sequences. Principle 2: Preserves monotonic mapping. Principle 4: Aligns with suffix transformation patterns.</principles_applied>
    </example>
    <example>
      <input>qi-si-pe-e</input>
      <output>ξιφεε, ξιφη</output>
      <principles_applied>Principle 1: q + s → ξ correspondence in this context. Principle 3: Maintains sparse one-to-one mapping. Principle 4: Vowel contraction pattern (e-e → η) follows established rules.</principles_applied>
    </example>
    <example>
      <input>re-u-ko-to-ro</input>
      <output>λευκτρον, λευκτροι</output>
      <principles_applied>Principle 1: r → λ distributional context. Principle 2: Preserves monotonic mapping. Principle 3: Maintains sparse mapping in consonant clusters. Principle 4: Aligns with known toponym patterns.</principles_applied>
    </example>
    <example>
      <input>qo-u-qo-ta</input>
      <output>βουβοτας</output>
      <principles_applied>Principle 1: q → β correspondence in appropriate contexts. Principle 2: Preserves monotonic mapping. Principle 4: Diphthong formation patterns (o-u → ου) align with documented cases.</principles_applied>
    </example>
    <example>
      <input>qe-qi-no-me-na</input>
      <output>γεγινωμενα</output>
      <principles_applied>Principle 1: q → γ correspondence in specific environments. Principle 2: Monotonic mapping preserved. Principle 4: Verbal form transformations follow established patterns.</principles_applied>
    </example>
    <example>
      <input>po-ro-wi-to-jo</input>
      <output>πλωfιστοιο</output>
      <principles_applied>Principle 2: Monotonic mapping preserved despite syllable reduction. Principle 3: Maintains structural sparsity. Principle 4: Genitive ending transformation follows documented patterns.</principles_applied>
    </example>
    <example>
      <input>po-ti-pi</input>
      <output>πορτις, πορτιφι</output>
      <principles_applied>Principle 1: Context-appropriate sound correspondences. Principle 3: One-to-one mapping maintained. Principle 4: Case ending transformation follows established patterns.</principles_applied>
    </example>
    <example>
      <input>a-ri-qa</input>
      <output>αρισβας</output>
      <principles_applied>Principle 1: q → β correspondence in this phonetic environment. Principle 2: Monotonic mapping preserved. Principle 4: Consonant addition follows patterns observed in anthroponyms.</principles_applied>
    </example>
    <example>
      <input>ko-no</input>
      <output>σκοινος</output>
      <principles_applied>Principle 1: Initial k- often corresponds to σκ- in Greek. Principle 2: Monotonic mapping preserved with initial consonant addition. Principle 4: Aligns with observed prosthetic consonant patterns.</principles_applied>
    </example>
  </examples>

  <evaluation_criteria>
    <criterion_1>
      Distributional Similarity Score (Principle 1):
      - High: Each character appears in similar phonetic environments in both words
      - Medium: Most characters show distributional similarity
      - Low: Significant distributional mismatches
      Only propose cognates with High or Medium scores.
    </criterion_1>
    <criterion_2>
      Monotonicity Assessment (Principle 2):
      - Verify that the relative ordering of phonemes is preserved
      - Check that any insertions/deletions follow documented phonological rules
      - Reject any mapping that requires unjustified reordering of characters
    </criterion_2>
    <criterion_3>
      Structural Sparsity Check (Principle 3):
      - Count the number of one-to-many or many-to-one mappings
      - Verify each non-one-to-one mapping has strong linguistic justification
      - Prefer cognates with higher percentage of one-to-one mappings
    </criterion_3>
    <criterion_4>
      Coverage Assessment (Principle 4):
      - Check how well the proposed mapping aligns with documented patterns
      - Verify that any novel correspondences have strong linguistic justification
      - Prioritize cognates that reinforce established correspondence patterns
    </criterion_4>
  </evaluation_criteria>

  <chain_of_thought_steps>
    <step1>
      Analyze the Linear B syllabograms and identify potential phoneme values.
      Create a mapping table showing possible correspondences, evaluating each against Principle 1 (Distributional Similarity).
    </step1>
    <step2>
      Apply Principle 2 (Monotonic Mapping) to:
      - Preserve the relative order of phonemes
      - Account for documented sound changes
      - Reject mappings requiring unjustified reordering
    </step2>
    <step3>
      Apply Principle 3 (Structural Sparsity) to:
      - Prioritize one-to-one phoneme correspondences
      - Justify any one-to-many or many-to-one mappings
      - Quantify the sparsity of each proposed mapping
    </step3>
    <step4>
      Apply Principle 4 (Cognate Overlap) to:
      - Check alignment with documented Linear B to Greek correspondence patterns
      - Evaluate how the proposed cognate reinforces or extends known patterns
      - Calculate a coverage score based on pattern alignment
    </step4>
    <step5>
      Rank candidate cognates by composite score across all four principles:
      - Distributional Similarity Score (25%)
      - Monotonicity Score (25%)
      - Structural Sparsity Score (25%)
      - Pattern Coverage Score (25%)
      Output at most 3 highest-ranked cognates, separated by commas.
    </step5>
  </chain_of_thought_steps>

  <quality_control>
    Final Quality Control:
    1. For each proposed cognate, explicitly verify all four principles
    2. Reject any cognate with a low score on any principle
    3. Limit output to maximum 3 highest-quality cognates
    4. If uncertain about quality, prefer to output fewer cognates
  </quality_control>

  <word>
    <linear_b_word>ko-no-so</linear_b_word>
    <completeness_level>MOSTLY_COMPLETE</completeness_level>
    <definition>knossos</definition>
    <proposed_cognates>
      <ancient_greek_word>κνωσονδε</ancient_greek_word>
      <matching_level>1.000</matching_level>
      <ancient_greek_word>κνωσος</ancient_greek_word>
      <matching_level>1.000</matching_level>
    </proposed_cognates>
    <entity_type>common name</entity_type>
  </word>

  <output_format>
    <output_description>
      format your output in json.
      Return an array of cognates (even if you are returning a single one) containing three fields:
      - the field cognate containing the cognate itself;
      - the field likelihood containing an estimate between 0 and 1 of how much you are sure the words are cognates, according to rules specified below in the section likelihood calibration and downweighting.
      - the field note containing the reasoning behind the choice, the applied phenomena and the passed and unpassed checks.
      For the cognate field, STRICTLY FOLLOW THE OUTPUT FORMATTING RULES
    </output_description>
    <output_formatting_rule_0>
      CHARACTER SET: use ONLY these characters: fhαβγδεζηθικλμνξοπρςστυφχψω. DO NOT use accents, breathing marks, subscript iota, or other diacritics.FORMAT: Output up to 3 cognates.
    </output_formatting_rule_0>

    <likelihood_calibration>
      <calibration_instructions>
        When estimating likelihood, use the following calibrated scale:
        0.75-1.00: Reserved ONLY for established cognates confirmed in scholarly literature with near certainty
        0.56-0.74: Good evidence but with some uncertainties or competing explanations
        0.40-0.55: Plausible connection with significant uncertainties
        0.00-0.39: Highly speculative with minimal supporting evidence
      </calibration_instructions>

      <automatic_downweighting>
        <factor_1>
          Reduce likelihood by 0.3 if the cognate requires:
          - Three or more non-trivial sound transformations
          - Any reordering of phonemes
          - Addition/deletion of multiple phonemes.
        </factor_1>
        <factor_2>
          Reduce likelihood by 0.2 if:
          - The word is rare in the corpus
          - The cognate proposal conflicts with existing scholarship
          - The semantic match requires significant stretching.
        </factor_2>
        <factor_3>
          Reduce likelihood by 0.4 if an unknown syllabogram appears in the linear B sequence. The likelihood of sequences with unknown syllabograms MUST ALWAYS BE LESS THAN 0.7, THIS IS A HARD CAP, NON-NEGOTIABLE. All unknown syllabograms are those starting with '*' and ending in a number, like '*19' or '*22'.
        </factor_3>
        <factor_4>
          Even if all principles are satisfied, without attestation in scholarly literature,
          no novel cognate proposal should receive likelihood above 0.85.
        </factor_4>
      </automatic_downweighting>

      <example_likelihoods>
        <high_example>
          ko-no-so → κνωσος: likelihood = 0.95
          (Near certainty: well-documented toponym with scholarly consensus)
        </high_example>
        <medium_high_example>
          a-ko-ra → αγορα: likelihood = 0.85
          (Good correspondence but with some phonological uncertainties)
        </medium_high_example>
        <medium_example>
          po-ti-ni-ja → ποτνια: likelihood = 0.75
          (Reasonable correspondence but requires several transformations)
        </medium_example>
        <lower_medium_example>
          a-re-ka-sa-da-ra → αλεξανδρα: likelihood = 0.65
          (Plausible but with multiple phonological adaptations and uncertainties)
        </lower_medium_example>
        <lower_medium_example>
          pe-ma → σπερμα: likelihood = 0.65
          (Speculative connection requiring multiple unsupported transformations)
        </lower_medium_example>
        <low_example>
          a-ka-ma-to → αγαμαι: likelihood = 0.5
          (Unclear transformation at the end of the word, now matching suffixes but probably a similar root)
        </low_example>
        <lower_example>
          *47-so-de → ασος: likelihood = 0.4
          (Unclear transformation in the suffix and unknown syllabogram in the sequence)
        </lower_example>
        <lower_example>
          *34-za-te-si → ζατεσις: likelihood = 0.3
          (Unknown syllabogram in the sequence, complete omission of a syllabogram for no good reason)
        </lower_example>
        <lower_example>
          *56-ko-qe → κλαγγη: likelihood = 0.2
          (Plenty of unclear transformation phenomena and unknown syllabogram in the sequence)
        </lower_example>
        <lower_example>
          *56-ni-di-ja → ανδρα: likelihood = 0.0
          (No clear connection or matching between the two words)
        </lower_example>
      </example_likelihoods>
    </likelihood_calibration>

    <critical_evaluation>
      <instruction>
        For each cognate proposal:
        - Actively search for weaknesses in the proposed cognate connection
        - List at least one specific uncertainty or alternative explanation
        - Consider what evidence would be needed to increase confidence
        - Apply higher standards for novel cognate proposals than for established ones
      </instruction>
      <uncertainty_prompt>
        Before finalizing your likelihood score, ask:
        - What scholarly sources would I need to confirm this?
        - What alternative cognates might explain this Linear B term?
        - What specific sound changes require justification?
        - Would experts in Mycenaean Greek agree with this analysis?
      </uncertainty_prompt>
    </critical_evaluation>
  </output_format>
</root>
\end{lstlisting}

\subsection{JSON output format}

Each response must be a JSON array of 1-3 objects.
Every object contains three fields:
\begin{itemize}[leftmargin=2em]
  \item \textbf{cognate}: string using only the allowed character set (Greek lowercase plus f/h placeholders),
  \item \textbf{likelihood}: real number in [0, 1] after calibration, caps, and penalties,
  \item \textbf{note}: brief justification that states passed/failed checks and key mappings.
\end{itemize}

\subsection{Example JSON output}

\begin{lstlisting}[language=json, breaklines=true, basicstyle=\ttfamily\small]
[
  {
    "cognate": "κνωσος",
    "likelihood": 0.95,
    "note": "This is a well-established cognate for the toponym Knossos, widely confirmed in scholarly literature. Principle 1 (Distributional Similarity): Initial 'ko-' corresponds to 'κ', medial 'no' to 'ν', and final 'so' to 'σ'. The vowels 'o' in Linear B frequently correspond to 'ω' or 'ο' in Ancient Greek. This aligns with observed patterns for toponyms. Principle 2 (Monotonic Mapping): The relative order of phonemes k-n-s is strictly preserved. The representation of the Greek consonant cluster 'κν' by 'ko-no' and the long vowel 'ω' by 'o' are standard Mycenaean writing conventions, not reordering. Principle 3 (Structural Sparsity): The mapping efficiently accounts for the limitations of the Linear B syllabary, maintaining a high degree of one-to-one phonemic correspondence for the root. Principle 4 (Cognate Overlap): This correspondence is a canonical example of a toponym and strongly reinforces established phonetic and orthographic patterns between Linear B and Ancient Greek."
  },
  {
    "cognate": "κνωσον",
    "likelihood": 0.90,
    "note": "Direct morphological variant (accusative singular) of 'κνωσος'. Principle 1: Mapping of 'ko-no-' to 'κνω-' is identical; 'so' aligns with 'σον' (accusative ending). Principle 2: Relative order preserved; nominative 'ς' → accusative 'ν' is regular inflection, not reordering. Principle 3: Efficient mapping for an inflection with largely one-to-one correspondences. Principle 4: Consistent with documented declensional patterns and other LB→GR transformations."
  }
]
\end{lstlisting}


\subsection{Reproducibility}

Model: gemini-2.5-flash.
Unless otherwise specified, the default generation settings were used; where non-default parameters (e.g., temperature, top-k, top-p) were set, they are reported alongside experimental results in the main text.
